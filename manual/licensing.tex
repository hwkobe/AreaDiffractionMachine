This program is released under the GNU General
Public License (GPL) version 2.\index{GNU}\index{GPL}
The license can e found at 
\url{http://www.gnu.org/licenses/old-licenses/gpl-2.0.html}.
For the most part, you are free to use and distribute 
this software. You are free to make any modifications to 
the code under the condition that any modifications are clearly 
stated and that the modificates are are released under the 
GPL version 2.

This software manual is also licensed under the GPL. 
This is a bit unconvential. I decided to do so after reading
several discussions online. Following Nathanael Nerode's 
article {\em Why You Shouldn't Use the GNU FDL}, I include in
this paper the clause ``for the purpose of applying the GPL to 
this document, I consider `source code' to refer to the texinfo 
source and `object code' to refer to the generated info, tex, 
dvi, [pdf] and postscript files.''\cite{Nerode03}

This program uses the software package
levmar for performing Levenberg-Marquardt nonlinear
least squares minimization.
\index{Least Squares Minimization}\index{Fitting}
It is released under the GPL. That package can be 
found at \url{http://www.ics.forth.gr/~lourakis/levmar/}.\cite{lourakis04LM}

This program uses the function get\_pck() from the CCP4 package
\index{CCP4}\index{get\_pck()}\index{Mar2300}\index{Mar3450}
DiffractionImage to uncompress Mar data. It
written by Dr. Claudio Klein.\cite{Klein95} 
This prgoram also uses the file marccd\_header.h form
the DiffractionImage packate. It
released under the GPL and can be found at
\url{http://www.ccp4.ac.uk/ccp4bin/viewcvs/ccp4/lib/DiffractionImage/}.\cite{DiffractionImage}

This program uses the EdfFile library (EdfFile.py) for reading and 
writing files of the ESRF Data Format. It is is part of the PyMCA 
library and is licensed under the GNU GPL version 2.\cite{PyMCA}

\index{Polygon Inclusion Testing}
This program also uses W. Randolph Frankin's pnpoly() 
function for performing a point inclusion in polygon test. 
This code can be found at
\url{http://www.ecse.rpi.edu/Homepages/wrf/Research/Short\_Notes/pnpoly.html}
We are in compliance with his software license which is 
reproduced below\cite{Franklin05}:
\begin{quotation}\em
Copyright (c) 1970-2003, Wm. Randolph Franklin

Permission is hereby granted, free of charge, to any person obtaining 
a copy of this software and associated documentation files (the 
``Software''), to deal in the Software without restriction, including 
without limitation the rights to use, copy, modify, merge, publish, 
distribute, sublicense, and/or sell copies of the Software, and to 
permit persons to whom the Software is furnished to do so, subject 
to the following conditions:

Redistributions of source code must retain the above copyright 
notice, this list of conditions and the following disclaimers.
Redistributions in binary form must reproduce the above copyright 
notice in the documentation and/or other materials provided with 
the distribution.
The name of W. Randolph Franklin may not be used to endorse or 
promote products derived from this Software without specific 
prior written permission.
THE SOFTWARE IS PROVIDED ``AS IS'', WITHOUT WARRANTY OF ANY KIND, 
EXPRESS OR IMPLIED, INCLUDING BUT NOT LIMITED TO THE WARRANTIES OF 
MERCHANTABILITY, FITNESS FOR A PARTICULAR PURPOSE AND 
NONINFRINGEMENT. IN NO EVENT SHALL THE AUTHORS OR COPYRIGHT HOLDERS 
BE LIABLE FOR ANY CLAIM, DAMAGES OR OTHER LIABILITY, WHETHER IN AN 
ACTION OF CONTRACT, TORT OR OTHERWISE, ARISING FROM, OUT OF OR IN 
CONNECTION WITH THE SOFTWARE OR THE USE OR OTHER DEALINGS IN THE 
SOFTWARE.
\end{quotation}

