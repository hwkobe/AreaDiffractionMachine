\section{The Integration Algorithm}

The purpose of performing an intensity integration is 
to create a plot of average intensity as a function
of either $Q$, $2\theta$, or $\chi$. The algorithm
for performing an intensity integration is pretty
straight foreward. In order to perform an intensity
integration, we must already know the calibration values
of the experiment for the paritcular image that will
be integrated. Than, a range and bin size for the
integration must be give. For example, you might
want to do a $Q-I$ integration from 2 to 5 with
100 bins. Whatever the range is, it 
must be specified before an integration is done.

The algorithm for performing the intensity integration
is as follows: loop over every pixel in the image. 
Add its intensity to the bin if it should be
in the bin based upon its value of $Q$, $2\theta$, or 
$\chi$. Remember that we need to use the calibration
values to calculate the correspondign $Q$, $2\theta$, and 
$\chi$ value using equation~\ref{ytermsydoubleprime}
\ref{xtermsxdoubleprime}, \ref{chitermsyx}, 
\ref{2thetatermsr}, and \ref{qterms2theta}.
After going through all the pixel, the bins then get averaged 
together. 

This program can constran the integration. 
This means that you can perform, for example,
a $Q$ integration of only those pixels with some
particular range of $\chi$ values. Or, you can
constaint your $\chi$ integration to on a particular
$Q$ range. This could be used, for example, to
perform a $\chi$ integration of only a particular
diffraction peak. The algorithm for performing
the constaint isn't any more complicated. You just
only bin a particular intensity value if it is
allowed by the constraint.

The program can allows for masking of pixels.
This complicates the algorithm a little bit further.
Whenever the program finds an intensity value
that should be masked (either because it is too 
large, too small, or in a polygon mask), it makes
sure not to bin that pixel and acts as though
it does not exist..

Finally, the program can perform a polarization 
correction to the integration. The polarization 
correction formula is
\begin{align}
    I&=Im/PF \\ 
    PF&=P(1 - (\sin(2\theta)\sin(\chi-90))^2) + 
    (1 - P)(1 - (\sin(2\theta)\cos(\chi-90))^2)
\end{align}
with $Im$ the measured intensity. If this
is selected, what happens
then is that all pixels have their intensity
value corrected by this formula before they
are binned. Note that the $2\theta$ and $\chi$
values correspond to the particular value
that is being corrected.

\section{Integrating with the Program}

All of the intensity integration is 
