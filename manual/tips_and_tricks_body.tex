\subsection{Calibration}

The Calibration tab lets you load into the program 
x-ray diffraction data and it lets you calibrate 
this data to determine the experimental parameters 
that characterize the diffraction experiment. You 
can load in diffraction data using the 
\gui{Data File:} input. The program 
recognizes \gui{mar2300}, \gui{mar3450}, 
\gui{mccd}, and \gui{tiff} data. To load 
multiple files at once, and work the with sum image, 
you need to go into the file part of the menu and 
select \gui{Open Multiple Files}.

Image Calibration is used to infer the experimental 
setup by analyzing standard diffraction data. This 
program characterizes a diffraction experiment 
with several parameters:
\begin{itemize}
    \item \gui{xc:}, \gui{yc:} - The $x$ 
and $y$ coordinates on the detector where the incoming 
x-ray beam would have hit the detector were there 
no sample in the way (in pixels).
    \item \gui{d:} - Distance from sample to 
        detector (in mm).
    \item \gui{E:} - Energy of the incoming beam (in eV).
    \item \gui{alpha:}, \gui{beta:} 2 orthogonal tilt 
        parameters of the detector (in degrees).
    \item \gui{R:} - The rotation of the detector 
        around the center (in degrees).
    \item \gui{pl:} - The pixel length of the image. 
        The width of one pixel (in microns).
    \item \gui{ph:} - The pixel height of the image. 
        The height of one pixel (in microns).
\end{itemize}

Before calibrating an image, three things must be done. 
First, you must load data taken of a standard calibration 
sample. Then, you must load a $Q$ data file with $Q$ values 
of peaks corresponding to the sample. Finally, the program 
needs an initial guess at the calibration values. This can 
be done with the \gui{Parameters} inputs. You can sometimes 
find some of the calibration data in the header of a 
diffraction file. If the header contains any of this data, 
you can push the \gui{Get From Header} button to put the 
header values into the inputs. Once an initial guess is 
given, the \gui{Do Fit} button will refine the calibration 
parameters and get the best guess at the real experimental 
parameters.

$2\theta$ is the angle of scatter of a particular beam of light 
which left the crystal and arrived at the detector at a certain 
point. $2\theta$ is a property of all points on the detector 
dependent on exactly how the diffraction experiment is set up. 
$Q$ is defined as $Q=4\pi \times \sin(2\theta/2)/\lambda$, the 
wavelength of the incoming beam. $D$ is defined as $2\pi/Q$.

If you prefer to deal with a photons' wavelength instead of its 
energy, you can go into the Calibration menu and switch from
\gui{Work in eV} to \gui{Work in Lambda}. If you do so, the
calibration parameter \gui{E:} will be replaced with
\gui{$\lambda$:} and the current value will be converted. 
The conversion between these is $E=hc/\lambda$.

The \gui{Q Data:} input lets you load in $Q$ data. The Format
of a $Q$ data file is a 2 column space or tab separated plain
text file. The first line must be \gui{Q    dQ}. The rest of
the file must be lines of $Q$ values followed by a $\Delta Q$
range. The $\Delta Q$ range gives the program a range in the
image where it can find the real diffraction peaks. You can
give a list of $D$ values instead. The first line must be of
the form \gui{D   dD} and this must be followed by lines of
$D$ and $\Delta D$. For the fit to work, no other diffraction
peaks in the image can fall within a $Q$ range. This program
stores many standard $Q$ files. To get at them, click the 
\gui{Calibration} option in the menu, then the \gui{Standard Q}
option, and pick the crystal that you are using.

You can modify the fit in a couple of ways. To do the fit,
the calibration algorithm will look a certain number of 
places in the image to find diffraction peak. It does by
running from the center of the image out and looking for
peaks in the $\Delta Q$ ranges. Once the program finds
these peaks, it can store the peak list and use the same
one when fitting again. This option can be changed using
the \gui{Use Old Peak List (if possible)?} selection. If
you want to change the number of peaks that the program
tries to find, you can use the \gui{Number of Chi?} input.
This tells the program how many of these radial slices from 
the center should be done. The \gui{Stddev?} input tell the 
program what ratio higher the peak must be then the standard
deviation of the background near the peak in order for the
peak to be considered real. The higher the value, the more
picky the program is about finding legitimate peaks.

If you know some of the experimental parameters exactly, you
can push the \gui{Fixed?} check box associated with that
variable and it will be not refined when fitting. The pixel 
length and pixel height are never refined.

To see how good the current calibration parameters are at
characterizing the loaded data, you can use the 
\gui{Draw Q Lines?} check box to make the program draw
on the diffraction image the lines of constant $Q$ given
by the $Q$ data file. You can also draw the $Q$ range 
specified in the $Q$ file using the \gui{Draw dQ Lines?} 
check box. To see what peaks the program found while doing the 
fit, you can select the \gui{Draw Peaks?} check box and the
peaks will be displayed on top of the diffraction image.

You can zoom into the diffraction image by left clicking
in the image, dragging your mouse, and then releasing.
You can zoom out by right clicking on the image. You can
pan across the image by shift clicking on the image and 
then dragging. You can make the image bigger or smaller 
by resizing the window.

In the file menu, you can use the \gui{Save Image} option
to save the current diffraction file in several popular
image formats. The image will be saved with the current
zoom level and any $Q$ lines, $\Delta Q$ lines, or peaks
drawn on top of it.

\subsection{Masking}

You can make the program ignore certain pixels when doing
diffraction analysis. This is all done on the \gui{Masking}
tab. To have the program ignore pixels above or below a 
certain value, use the threshold masking option.
You can have the program ignore all pixels that are larger 
then a certain value by selecting 
\gui{Do Greater Than Mask?} and putting the particular value
into the \gui{(Pixels Can't Be) Greater Than Mask:} input.
You can have the program ignore all pixels less than a certain
value with the corresponding \gui{Do Less Than Mask?}
and the \gui{(Pixel's Can't Be) Less Than Mask:} input.
Once you set the threshold, these any pixel either too large
or too small will be displayed on the diffraction and cake
displays as a different color. That color is user 
specified by the color inputs next to the checkboxes. 
When you do an intensity integration, these pixels are 
not used in the averages.

You can also have the program mask certain whole areas 
of the image if those pixels are undesirable for some 
reason. If you wish to do so, select the 
\gui{Do Polygon Mask?} option. To draw the image on the
screen that you do not want, click \gui{Add Polygon} and
then then left click on the diffraction image for each 
of the vertices of the polygon that you want to mask out.
To draw the final vertex and finish the polygon, just
use right click. To remove any polygon, push the 
\gui{Remove Polygon} button and click on the polygon that
you want to get rid of. You can remove all the polygons
in the program using the \gui{Clear Mask} button, save
all the polygons in the program to a file using the
\gui{Save Mask} button, and load saved polygons using
the \gui{Load Mask} button.

\subsection{Cake}

A caked image is a plot of diffraction data in $Q$, 
$\chi$ space. $\chi$ is a measure of the angle around 
what would be the incoming x-ray beam if the beam was 
not scattered. By convention, $\chi$ is equal to 0
degrees to the right of the pixel center of the image 
and increases while going counterclockwise. To cake the
data, you need to give the program a range and bin size 
in both $Q$ and $\chi$. When you push \gui{Do Cake}, the
program will present a new window with the caked data in
it. You can interact with the cake image just like with
the diffraction image described above. If you have one of
the \gui{Draw Q Lines}, or \gui{Draw dQ Lines} or 
\gui{Draw Peaks} boxes selected from the \gui{Calibration} 
tab, these will also be drawn on the cake image. In particular,
the $Q$ and $\Delta Q$ lines are just vertical lines on the
caked image. You can save the caked data as plain text 
using the \gui{Save Data} button. You can save the caked 
image as a popular image format using the \gui{Save Image}
button. The image will have the $Q$ lines and peaks saved
in the image if that option is selected when saving.

You can also apply a polarization correction to the caked
data using the \gui{Do Polarization Correction?} check box
and the \gui{P?} input. The formula to calculate a 
polarization correction is:
\begin{eqnarray}
    I&=&Im/PF \\ 
    PF&=&P(1 - (\sin(2\theta)\sin(\chi-90))^2) + 
    (1 - P)(1 - (\sin(2\theta)\cos(\chi-90))^2)
\end{eqnarray}
with $Im$ the measured intensity. There is a convenient
feature called AutoCake which picks a range of $Q$ and $\chi$ 
values and then cakes that range. AutoCake will pick the
smallest cake range so that the whole image shows up in
the cake. It will pick the bin size so that each pixel 
displayed on the screen corresponds to a single bin.

\subsection{Integrate}

Intensity integration lets you calculate average intensity
as a function of $Q$, $\chi$, or $2\theta$. By default, the
option is to integrate in $Q$ or $\chi$. If you want to 
integrate in $2\theta$, you must click on the integrate 
option of the menu and select \gui{Work in 2theta} instead
of \gui{Work in Q}. To integrate intensity, you need to
give a lower value, an upper value, and a bin size. With 
these inputs, the \gui{Integrate} button will perform the
integration. This data will be plotted in a new window.
By default, the integration will be over all possible values
of the other independent variable. If you integrate in $Q$,
it will be for all $\chi$. This can be changed by using the
constrain checkboxes. For example, to integrate in $Q$ only
for the $\chi$ values between 0 and 90, you would select the
\gui{Constraint With Range on Right?} option and the put in
the \gui{Chi Lower?} input the value 0 and the \gui{Chi Upper?}
input the value 90.  When you then pushed the integrate button, 
the program would apply the proper constraints.

You can apply the same sort of polarization correcting when 
integrating. You can also save integration data using the
\gui{Save Data} button. The output file will be many rows
of two column plain text data. 

\subsection{Macro}

If you want to perform a complicated task many times, the
easiest thing to do is to record a macro of all the commands
and then run the macro instead of doing everything by hand.
To record a macro, go into the Macro part of the menu and
select \gui{Start Record Macro}. After doing the tasks you
want recorded, push \gui{Stop Record Macro} and can save 
these macro commands to a text file. To run a macro file,
you can go into the Macro part of the menu and select
\gui{Run Saved Macro}. 

By editing macro files by hand, you can make them much more
useful. Most macro commands are just the name of the GUI item
possibly followed by whatever that thing would want (like a 
filename or number). The macro command to load a diffraction 
file is \macroline{Data File:} followed by a line with a
filename such as \macroline{C:/somefolder/thefile.mar3450}.
The filename line can be replaced by a list of filenames,
a directory containing diffraction data, or some combination
of both. The program will run the subsequent macro lines on
every file given in the list and on all diffraction files found 
in any folders given in the list. The loop over diffraction
files will end with a subsequent \macroline{Data File:} line,
a \macroline{END LOOP} line, or the end of the macro file.

When lopping over diffraction files, there is a special 
markup which makes it easy to save files into the right 
place. This can be done with the special markup names
\macroline{BASENAME} and \macroline{FILENAME}. Whenever the
program runs across \macroline{BASENAME} in the macro file,
that string will be replaced with the current base of the 
diffraction file that has been loaded. Similarly, 
\macroline{FILENAME} will get replaced with the filename of
the current diffraction file. In our previous example,
\macroline{BASENAME} would get replaced with
\macroline{C:/somefolder} and \macroline{PATHNAME} would get 
replaced with \macroline{thefile}. You could recreate the 
original file name with \macroline{PATHNAME/FILENAME.mar3450}.
This markup can be used to save things with useful names. For
example, the macro line \macroline{Save Integration Data}
followed by the line
\macroline{PATHNAME/FILENAME\_q\_i\_integration.dat} would, for
our previously loaded file, save the integrated intensity data
as \macroline{C:/somefolder/thefile\_q\_i\_integration.dat}. For
all other diffraction data being looped over, the diffraction data
would get saved next to the corresponding data file.
