\chapter{Tips and Tricks}

\section{Calibration}

The \gui{Calibration} tab can also be used to load
diffraction data into the program. The tab can
be used to calibrate diffraction data to determine the 
parameters that characterize the experiment. 
Diffraction data can be loaded using the \gui{Data File:} input. 
The program recognizes \macroline{mar2300}, \macroline{mar3450}, 
\index{Mar2300}\index{Mar3450}\index{MarCCD}\index{Tiff}
\macroline{mccd}, \macroline{tiff}, and \macroline{edf} data. 
Multiple files can be loaded into the program at the same
time using the file selector and the sum image is loaded.

This program characterizes a diffraction experiment 
according to the parameters:
\index{Calibration Parameters}
\index{X Center}\index{Y Center}
\index{Detector Distance}\index{$\alpha$}
\index{$\beta$}\index{Rotation}\index{Pixel Scale}
\begin{itemize}
    \item \gui{xc:}, \gui{yc:} - the $(x,y)$ pixel coordinate 
    on the detector where the incoming 
    x-ray beam would have hit the detector were there 
    no sample in the way (in pixels).
    \item \gui{d:} -- the distance from the sample to 
    the detector (in mm).
    \item \gui{E:} -- the energy of the incoming beam (in eV).
    \item \gui{alpha:}, \gui{beta:} \gui{R} -- 3 rotations of 
    the detector (in degrees).
    \item \gui{pl:} -- the pixel length of the detector - 
    the width of one pixel (in microns).
    \item \gui{ph:} -- The pixel height of the detector -
    the height of one pixel (in microns).
\end{itemize}
Before calibrating an image, three things must be done. 
First, the diffraction data must be loaded.
Second, a $Q$ data file with standard $Q$ values for the
sample must be loaded.
Third, an initial guess of the calibration parameters must be
loaded. A guess at the calibration parameters can sometimes
be found in the header of the diffraction file. These values
can be loaded into the program using the \gui{Get From Header} 
button.  The \gui{Do Fit} button will perform the calibration 
and find a best guess at the real experimental parameters.

The \gui{Work in Lambda} option in the \gui{Calibration} menu can
be used to make the program work with the x-ray's
wavelength instead of its energy. The calibration parameter 
\gui{$\lambda$:} will be used instead of \gui{E:}.

The \gui{Q Data:} input will load into the program standard $Q$ data files. 
This program stores several standard $Q$ files that can be selected
through the \gui{Standard Q} option of the \gui{Calibration} menu.

The calibration algorithm can be modified in a couple of ways.
The program finds peaks in the diffraction data by 
running from the center of the image out. 
The number of radial slices that the program uses
can be set with the \gui{Number of Chi?} input.
The \gui{Stddev?} input 
tells the program what ratio higher a peak must be than the 
standard deviation of the background near it for the
peak to be considered real. The higher the value, the more
picky the program is about finding legitimate peaks.

If some of the experimental parameters are known exactly, 
the \gui{Fixed?} check box associated with that variable 
will stop it from being refined when fitting. The pixel 
length and pixel height are never refined. 

To see how good the current calibration parameters are, 
the \gui{Draw Q Lines?} 
check box will make the program draw 
on the diffraction image lines of constant $Q$ specified
by the $Q$ data file. The $\Delta Q$ ranges specified in 
the $Q$ data file can be drawn using the \gui{Draw dQ Lines?} 
check box. The diffraction peaks that were found while
calibrating can be displayed using the \gui{Draw Peaks?} check box. 

The diffraction image can be zoomed into by left clicking
on the image, dragging the mouse, and then releasing.
The image can be unzoomed by right clicking on the image. 
The image can be panned across by shift clicking on the image 
and dragging. The image can be made bigger or smaller by 
resizing the window.

In the \gui{File} menu, the \gui{Save Image} option can be used 
to save the current diffraction file in several popular
image formats. The image will be saved with the current
zoom level and any $Q$ lines, $\Delta Q$ lines, peaks, or
masks drawn on it.

\section{Masking}
\index{Pixel Masking}

The program can ignore certain pixels in an image
when performing diffraction analysis. This can be done
using the \gui{Masking} tab.  Threshold masking can be 
used to ignore pixels above or below a certain value.
All pixels larger than a certain value can be masked 
using the \gui{Do Greater Than Mask?} check box
and specifying the value in the 
\gui{(Pixels Can't Be) Greater Than Mask:} input.
All pixels less than a certain value can be ignored 
using the \gui{Do Less Than Mask?} check box 
and specifying the value in the 
\gui{(Pixels Can't Be) Less Than Mask:} input.
The overloaded or underloaded pixels will show up
as a different color on the diffraction and cake images.
This color can be changed using the \gui{Color} buttons
near the other inputs.
When a threshold mask is applied, masked pixels
will not be used during an intensity integration.

The program can mask areas of the diffraction image
using polygon masks. The \gui{Do Polygon Mask?} check box 
will enable this. Any masks in the program
will be displayed over the diffraction data and cake images.
Any masked pixels will not be used during an intensity
integration. The \gui{Add Polygon} button can be used to 
draw new polygon masks. To draw a mask, simply push
the button, left click all the
nodes on the diffraction image except the last one, and
then right click the final node. 
The \gui{Remove Polygon} button can be used to 
remove polygons from the diffraction image. Simply push
the button and then click on the polygon that should be
removed. The \gui{Clear Mask} button will remove all
the polygons from the program. The \gui{Save Mask} button
and the \gui{Load Mask} button will save and load
polygons to and from a file.

\section{Caking}
\index{$\chi$}\index{$Q$}\index{Caking}

A cake is a plot of diffraction data in $Q$ vs.
$\chi$ space. $\chi$ is a measure of the angle around 
the incoming x-ray beam. By convention, $\chi$ is equal to 
$0\degrees$ to the right of the center of the image and
increases in a counterclockwise direction. 
The program needs to know a range and bin size in $Q$ and
$\chi$ in order to make a caked plot. The \gui{Do Cake} button
will create a cake and open a new window with the caked data in
it. The caked window can be interacted with just like
the diffraction window. There is a button called \gui{AutoCake} that 
picks a cake range with the whole diffraction image in it 
and then cakes the data. Any $Q$ lines, $\Delta Q$ lines,
and peaks that are drawn on the diffraction image
will also be drawn on the caked image.
The \gui{Save Data} button will save the caked data
to a file as plain text. The \gui{Save Image} button will save the
caked plot as a popular image format with
any $Q$ lines or peaks that are drawn on 
the caked plot saved on top of it.

The \gui{Do Polarization Correction?} button will apply a 
polarization correction to the caked plot. The polarization
of the incoming beam can be specified with the 
\gui{P?} input. The formula for calculating the 
polarization correction is
\begin{eqnarray}
    I&=&Im/PF \\ 
    PF&=&P(1 - (\sin(2\theta)\sin(\chi-90))^2) + 
    (1 - P)(1 - (\sin(2\theta)\cos(\chi-90))^2)
\end{eqnarray}
with $Im$ the measured intensity. 


\section{Integrate}
\index{Intensity Integration}

An intensity integration is a plot of average intensity
vs. $Q$, $\chi$, or $2\theta$. By default, the options 
available are to integrate in $Q$ or $\chi$. The 
\gui{Work in 2theta} option in the menu bar can be used 
to make the program integrate in $2\theta$ instead of $Q$.  
The program needs to know a range (both a lower and upper value)
and a bin size in order to perform an intensity integration.
When these values have been loaded, the \gui{Integrate} button will 
perform the integration. A new window will open with a plot of data.
By default, the integration will be over all 
possible values of the other variable. This can be changed using the
constraint check boxes. 
For example, selecting the \gui{Constraint With Range on Right?}
check box and setting the \gui{Chi Lower?} input to 0 and the
\gui{Chi Upper?} into to 90 will cause the integration in $Q$
to be only of pixel's with $\chi$ between 0 and 90. 

A polarization correction can be applied during
an intensity integration. The \gui{Save Data} button can be used
to save the intensity data to a file as two column ASCII. 

\section{Macro}\index{Macros}

Macros can be used to speed up data analysis. 
The \gui{Start Record Macro} option in the \gui{Macro} menu
will make the program start recording a macro. After the desired
tasks have been recorded, the \gui{Stop Record Macro} option will
stop the recording and save the commands to a file. The
\gui{Run Saved Macro} option will run a macro file.

Small edits to a macro file can make them much more versatile. 
Most macro commands are just the name of the GUI item
possibly followed by whatever the GUI would want (such as a 
filename or a number). The macro command to load a diffraction 
file is \macroline{Data File:} followed by a line with 
a filename. It can also be followed by a list of filenames,
a directory containing diffraction data, or some combination
of each. The program will run the subsequent macro on
every file in the list and all diffraction files 
in folders in the list. The loop will end with a subsequent 
\macroline{Data File:} line, a \macroline{END LOOP} line, or 
the end of the macro file.

When looping over diffraction files, there is special 
markup which makes it easy to save files in a loop with 
useful names. Whenever the
program finds \macroline{BASENAME} in a macro file,
it will be replaced with the path of the current
diffraction file that has been loaded. Similarly, 
\macroline{FILENAME} will be replaced with the filename of
the current diffraction file. A
diffraction file with the extension \macroline{.mar3450} could be 
recreate with the command \macroline{PATHNAME/FILENAME.mar3450}.
An example of the use of this would be the macro line 
\macroline{Save Integration Data} followed by the line
\macroline{PATHNAME/FILENAME\_int.dat}. The program would always
save the intensity integrated data right next to the diffraction 
file with the same filename but ending with \macroline{\_int.dat}.
